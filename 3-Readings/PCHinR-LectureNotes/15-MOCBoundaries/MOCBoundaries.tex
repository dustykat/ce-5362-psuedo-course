\section{Using Method of Characteristics for Boundary Conditions}
The method of characteristics is a technique used to find boundary conditions for the finite-difference model(s).
The readings for this chapter (on the server) contain several examples of the method of characteristics to establish boundary values for the model

Starting again with continuity and momentum

\begin{equation}
\frac{\partial y}{\partial t} = -\frac{A}{B}\frac{\partial V}{\partial x}-V\frac{\partial y}{\partial x}
\end{equation}

\begin{equation}
\frac{\partial V}{\partial t} = g(S_0-S_f)-V\frac{\partial V}{\partial x}-g\frac{\partial y}{\partial x}
\end{equation}

multiply the continuity equation by a term $\lambda$, them add the result to the momentum equation to obtain

\begin{equation}
\lambda \frac{\partial y}{\partial t} + \lambda \frac{A}{B}\frac{\partial V}{\partial x} + 
\lambda V\frac{\partial y}{\partial x} + g\frac{\partial y}{\partial x} + 
\frac{\partial V}{\partial t} + V\frac{\partial V}{\partial x}= g(S_0-S_f)
\end{equation}

Define the $\lambda$ as
\begin{equation}
V + \lambda \frac{A}{B} = \frac{dx}{dt} = V + \frac{g}{\lambda}
\end{equation}

where $\frac{dx}{dt}$ is from the total derivative
\begin{equation}
\frac{dF}{dt} = \frac{\partial F}{\partial t} + \frac{\partial F}{\partial x}\frac{\partial x}{\partial t}
\end{equation}

[ need sketchs to define + and - characteristics then extrapolate back to the boundary.]
[ for Summer 2017 - use the boundary conditions in the code without explain]
